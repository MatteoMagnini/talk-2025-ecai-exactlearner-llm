% !TeX spellcheck = en_GB
%%%%%%%%%%%%%%%%%%%%%%%%%%%%%%%%%%%%%%%%%%%%%%%%%%%%%%%%%%%%%%%%%%%%%%%%%%%%%%%%
%\documentclass[handout]{beamer}\mode<handout>{\usetheme{default}}
%
%\documentclass[presentation]{beamer}\mode<presentation>{\usetheme{blackAMSBolognaFC}}
\documentclass[handout]{beamer}\mode<handout>{\usetheme{AMSBolognaFC}}
% \setbeamertemplate{bibliography item}{\insertbiblabel}
%%%%%%%%%%%%%%%%%%%%%%%%%%%%%%%%%%%%%%%%%%%%%%%%%%%%%%%%%%%%%%%%%%%%%%%%%%%%%%%%
\usepackage[english]{babel}
\usepackage[utf8]{inputenc}
%
\usepackage{magnini-ecai-2025-talk}
%%%%%%%%%%%%%%%%%%%%%%%%%%%%%%%%%%%%%%%%%%%%%%%%%%%%%%%%%%%%%%%%%%%%%%%%%%%%%%%%
\title[Learning \EL Terminologies from LLMs]{
    % same title of the presented paper
    Actively Learning \EL Terminologies
    \\
    from Large Language Models
}
%
% \subtitle{Extended Abstract}
%
% same authors order of the presented paper
\author[Magnini et al.]{
	\emph{Matteo Magnini}$^{*}$ % empth the presenting author
	\and 
	Riccardo Squarcialupi$^{*}$
	\\
	Martin T. Sterri$^{\dagger}$
	\and
	Ana Ozaki$^{\dagger,\ddagger}$
}
%
\institute[UniBo, Uni*]{
    $^{*}$%Department of Computer Science and Engineering
    %\\
    \textsc{Alma Mater Studiorum} -- University of Bologna
    \\
    \texttt{
        \emph{matteo.magnini}@unibo.it, riccard.squarcialupi@studio.unibo.it
    }
    \vspace{.3cm}
    \\
    $^{\dagger}$University of Bergen
    \\
    \texttt{
        martin.sterri@student.uib.no, ana.ozaki@uib.no
    }
    \vspace{.3cm}
    \\
    $^{\ddagger}$University of Oslo
    \\
    \texttt{
        anaoz@ifi.uio.no
    }
}
%
\date[ECAI, 2025]{
	The European Conference on Artificial Intelligence (ECAI 2025)
	\\
	27 October, 2025, Bologna
}
%%%%%%%%%%%%%%%%%%%%%%%%%%%%%%%%%%%%%%%%%%%%%%%%%%%%%%%%%%%%%%%%%%%%%%%%%%%%%%%%
\AtBeginSection[]
{
%\\\\\\\\\\\\\\\\\\\\\
\begin{frame}<beamer>[c,noframenumbering]
\frametitle{Next in Line\ldots}
\tableofcontents[sectionstyle=show/shaded,subsectionstyle=hide]
\end{frame}
%\\\\\\\\\\\\\\\\\\\\\
}
\AtBeginSubsection[]
{
%\\\\\\\\\\\\\\\\\\\\\
\begin{frame}<beamer>[shrink,noframenumbering]
    \frametitle{Focus on\ldots}
	\mbox{~}
	\tableofcontents[currentsubsection,sectionstyle=shaded,subsectionstyle=show/shaded]
	\mbox{~}
\end{frame}
%\\\\\\\\\\\\\\\\\\\\\
}
%%%%%%%%%%%%%%%%%%%%%%%%%%%%%%%%%%%%%%%%%%%%%%%%%%%%%%%%%%%%%%%%%%%%%%%%%%%%%%%%
\begin{document}
%%%%%%%%%%%%%%%%%%%%%%%%%%%%%%%%%%%%%%%%%%%%%%%%%%%%%%%%%%%%%%%%%%%%%%%%%%%%%%%%

%\\\\\\\\\\\\\\\\\\\\\
\frame{\titlepage}
%\\\\\\\\\\\\\\\\\\\\\

%===============================================================================
\section{Motivation \& Context}
%===============================================================================

%\\\\\\\\\\\\\\\\\\\\\
\begin{frame}[c, allowframebreaks]{Context}

    \vfill
    
    The \alert{active learning} framework:
    %
    \vfill
    %
    \begin{itemize}
        \item a \alert{learner} attempts to learn some kind of \alert{knowledge}
        %
        \item by posing questions to a \alert{teacher}

        \vfill

        \item questions made by the learner are
        %
        \begin{itemize}
            \item \alert{membership} queries $\rightarrow$ ask whether \alert{concept inclusions} are true or false
            %
            \item \alert{equivalence} queries $\rightarrow$ ask whether the idea od the learner about the knowledge of the teacher is correct or not
        \end{itemize}
        
        \vfill
        
    \end{itemize}

    \framebreak

    \begin{figure}
        \centering
        \includegraphics[width=0.8\textwidth]{figures/queries-example}
        \caption{Example of membership and equivalence queries}
        \label{}
    \end{figure}

    We want to use \alert{Large Language Models} (LLMs) as teachers in the \alert{Angluin}'s exact learning framework \ccite{DBLP:journals/ml/Angluin87}.

\end{frame}
%\\\\\\\\\\\\\\\\\\\\\

%\\\\\\\\\\\\\\\\\\\\\
\begin{frame}[c]{Motivation}
    Some motivations for our work:
    %
    \vfill
    %
    \begin{itemize}
        \item to the best of our knowledge, the only implementation of the Angluin's exact learning framework uses a \alert{synthetic teacher} \ccite{DBLP:conf/kr/DuarteKO18}
        %
        \item ontology construction is a costly and time-consuming task that requires domain experts
        %
        \item arguably, a boring and repetitive task for humans
        %
        \item with LLMs as teachers, we can \alert{automate} the process of ontology construction
        %
        \item with Angluin's framework, we build ontologies in a systematic way
        %
    \end{itemize}
\end{frame}
%\\\\\\\\\\\\\\\\\\\\\


%===============================================================================
\section{Design}
%===============================================================================

%\\\\\\\\\\\\\\\\\\\\\
\begin{frame}[c, allowframebreaks]
\frametitle{Algorithm}

    \begin{figure}
        \centering
        \includegraphics[width=0.8\textwidth]{figures/algorithm}
        \caption{Overview of the exact learning algorithm}
        \label{fig:algorithm}
    \end{figure}

    \framebreak

%    \vfill

    Equivalence query are \alert{symulated} via random \alert{sampling}.
    %
    The algorithm checks if the classification of the examples match with the information in the hypothesis:

    \begin{itemize}
        %
        \item true inclusions must be \alert{logical consequences}
        %
        \item false ones must not
        %
    \end{itemize}

%    \vfill

    If the hypothesis fits the classification of the concept inclusions, learning stops.
    Otherwise, the inclusion not fitting the hypothesis is used as a \alert{counterexample}.

%    \vfill

    \framebreak

    The sampling-based simulation can yield \alert{PAC} \ccite{DBLP:journals/cacm/Valiant84} guarantees when the sample size

    \begin{equation*}
        \lvert S \rvert \geq \frac{\ln\left(\lvert H \rvert / \gamma\right)}{\epsilon}
    \end{equation*}

    is computed from the hypothesis space $H$ (\EL terminologies of bounded structure) and parameters $\epsilon$ (error) and $\gamma$ (confidence).


\end{frame}
%\\\\\\\\\\\\\\\\\\\\\

%\\\\\\\\\\\\\\\\\\\\\
\begin{frame}[c, allowframebreaks]
\frametitle{Learner's operations}

    When the teacher replies with a counterexample, the learner before adding it to the hypothesis \alert{processes} it.
    The learner performs operations, that use membership queries, in order to \alert{maximise} how informative the concept inclusions are and also to \alert{minimise} their size.

    \begin{itemize}
        \item Decompose Left
        \item Decompose Right
        \item Merging
        \item Branching
        \item Saturation
        \item Desaturation
    \end{itemize}

    \framebreak

    Decompose Right

    \begin{columns}[c]
    \column{0.55\textwidth}
    \vspace{-1cm}
    \begin{equation*}
    \begin{aligned}
        T &= \{A \sqsubseteq \exists r.\top,\; B \sqsubseteq \exists r.\top,\; A \sqsubseteq B\} \\
        H &= \{A \sqsubseteq B\} \\
        \text{C} &= A \sqsubseteq
            \textcolor{blue}{B \sqcap}
            \textcolor{orange}{\exists s.\top}
            \textcolor{blue}{\sqcap}
            \textcolor{green}{\exists r.\top}
        \\[0.8em]
        &\Downarrow \\[0.8em]
        \text{C} &= \textcolor{blue}{B} \sqsubseteq
            \textcolor{orange}{\exists s.\top}
    \end{aligned}
    \end{equation*}

    \column{0.45\textwidth}
    \centering

    \vspace{1cm}

    \begin{tikzpicture}[
      every node/.style={
        draw, rounded corners=4pt,
        minimum width=1.6cm, minimum height=0.8cm,
        align=center, thick
      },
      root/.style={draw=blue},
      leftchild/.style={draw=orange!90!black},
      rightchild/.style={draw=green!60!black},
      arrow/.style={->, very thick, shorten >=2pt, shorten <=2pt}
    ]
        \node[root] (root) {\{$B$\}};
        \node[leftchild, below left=0.7cm and 0.1cm of root] (left) {\{\}};
        \node[rightchild, below right=0.7cm and 0.1cm of root] (right) {\{\}};
        \draw[arrow, orange!90!black] (root) -- (left)
            node[midway, above left=-0.1cm, text=black, font=\small, fill=none, draw=none] {$s$};
        \draw[arrow, green!60!black] (root) -- (right)
            node[midway, above right=-0.1cm, text=black, font=\small, fill=none, draw=none] {$r$};
    \end{tikzpicture}

    \vspace{1cm}

    \begin{tikzpicture}[
      every node/.style={
        draw, rounded corners=4pt,
        minimum width=1.6cm, minimum height=0.8cm,
        align=center, thick
      },
      root/.style={draw=black},
      leftchild/.style={draw=orange!90!black},
      rightchild/.style={draw=black},
      arrow/.style={->, very thick, shorten >=2pt, shorten <=2pt}
    ]
        \node[root] (root) {\{$B$\}};
        \node[leftchild, below left=0.7cm and 0.1cm of root] (left) {\{\}};
        \node[rightchild, below right=0.7cm and 0.1cm of root] (right) {\{\}};
        \draw[arrow, orange!90!black] (root) -- (left)
            node[midway, above left=-0.1cm, text=black, font=\small, fill=none, draw=none] {$s$};
        \draw[arrow, black] (root) -- (right)
            node[midway, above right=-0.1cm, text=black, font=\small, fill=none, draw=none] {$r$};
    \end{tikzpicture}

    \end{columns}
\end{frame}

%\\\\\\\\\\\\\\\\\\\\\

\section{Conclusions \& future works}

%\\\\\\\\\\\\\\\\\\\\\
\begin{frame}%[allowframebreaks]
\frametitle{Conclusions \& future works}

\begin{block}{Summing up}
    Summarise the most relevant contributions of this study:
    %
    \begin{itemize}
        \item conclusion 1
        \item conclusion 2
        \item conclusion 3
    \end{itemize}
\end{block}

\begin{exampleblock}{Future works}
    Sketch some future research directions
    %
    \begin{itemize}
        \item future work 1
        \item future work 2
    \end{itemize}
\end{exampleblock}

(may be split into 2 slides)

\end{frame}
%\\\\\\\\\\\\\\\\\\\\\

%===============================================================================
\section*{}
%===============================================================================
\frame{\titlepage}

%===============================================================================
\section*{\bibname}
%===============================================================================

\setbeamertemplate{page number in head/foot}{}
%\\\\\\\\\\\\\\\\\\\\\
\begin{frame}[t,allowframebreaks,noframenumbering]\frametitle{\refname}
% \begin{frame}[c]\frametitle{\refname}
	\footnotesize
%	\scriptsize
    \bibliographystyle{apalike-AMS}
    % \bibliographystyle{plain}
	\bibliography{magnini-ecai-2025-talk}
\end{frame}
%\\\\\\\\\\\\\\\\\\\\\

%%%%%%%%%%%%%%%%%%%%%%%%%%%%%%%%%%%%%%%%%%%%%%%%%%%%%%%%%%%%%%%%%%%%%%%%%%%%%%%%
\end{document}
%%%%%%%%%%%%%%%%%%%%%%%%%%%%%%%%%%%%%%%%%%%%%%%%%%%%%%%%%%%%%%%%%%%%%%%%%%%%%%%%
